\PassOptionsToPackage{unicode=true}{hyperref} % options for packages loaded elsewhere
\PassOptionsToPackage{hyphens}{url}
%
\documentclass[]{article}
\usepackage{lmodern}
\usepackage{amssymb,amsmath}
\usepackage{ifxetex,ifluatex}
\usepackage{fixltx2e} % provides \textsubscript
\ifnum 0\ifxetex 1\fi\ifluatex 1\fi=0 % if pdftex
  \usepackage[T1]{fontenc}
  \usepackage[utf8]{inputenc}
  \usepackage{textcomp} % provides euro and other symbols
\else % if luatex or xelatex
  \usepackage{unicode-math}
  \defaultfontfeatures{Ligatures=TeX,Scale=MatchLowercase}
\fi
% use upquote if available, for straight quotes in verbatim environments
\IfFileExists{upquote.sty}{\usepackage{upquote}}{}
% use microtype if available
\IfFileExists{microtype.sty}{%
\usepackage[]{microtype}
\UseMicrotypeSet[protrusion]{basicmath} % disable protrusion for tt fonts
}{}
\IfFileExists{parskip.sty}{%
\usepackage{parskip}
}{% else
\setlength{\parindent}{0pt}
\setlength{\parskip}{6pt plus 2pt minus 1pt}
}
\usepackage{hyperref}
\hypersetup{
            pdfborder={0 0 0},
            breaklinks=true}
\urlstyle{same}  % don't use monospace font for urls
\usepackage{color}
\usepackage{fancyvrb}
\newcommand{\VerbBar}{|}
\newcommand{\VERB}{\Verb[commandchars=\\\{\}]}
\DefineVerbatimEnvironment{Highlighting}{Verbatim}{commandchars=\\\{\}}
% Add ',fontsize=\small' for more characters per line
\newenvironment{Shaded}{}{}
\newcommand{\AlertTok}[1]{\textcolor[rgb]{1.00,0.00,0.00}{\textbf{#1}}}
\newcommand{\AnnotationTok}[1]{\textcolor[rgb]{0.38,0.63,0.69}{\textbf{\textit{#1}}}}
\newcommand{\AttributeTok}[1]{\textcolor[rgb]{0.49,0.56,0.16}{#1}}
\newcommand{\BaseNTok}[1]{\textcolor[rgb]{0.25,0.63,0.44}{#1}}
\newcommand{\BuiltInTok}[1]{#1}
\newcommand{\CharTok}[1]{\textcolor[rgb]{0.25,0.44,0.63}{#1}}
\newcommand{\CommentTok}[1]{\textcolor[rgb]{0.38,0.63,0.69}{\textit{#1}}}
\newcommand{\CommentVarTok}[1]{\textcolor[rgb]{0.38,0.63,0.69}{\textbf{\textit{#1}}}}
\newcommand{\ConstantTok}[1]{\textcolor[rgb]{0.53,0.00,0.00}{#1}}
\newcommand{\ControlFlowTok}[1]{\textcolor[rgb]{0.00,0.44,0.13}{\textbf{#1}}}
\newcommand{\DataTypeTok}[1]{\textcolor[rgb]{0.56,0.13,0.00}{#1}}
\newcommand{\DecValTok}[1]{\textcolor[rgb]{0.25,0.63,0.44}{#1}}
\newcommand{\DocumentationTok}[1]{\textcolor[rgb]{0.73,0.13,0.13}{\textit{#1}}}
\newcommand{\ErrorTok}[1]{\textcolor[rgb]{1.00,0.00,0.00}{\textbf{#1}}}
\newcommand{\ExtensionTok}[1]{#1}
\newcommand{\FloatTok}[1]{\textcolor[rgb]{0.25,0.63,0.44}{#1}}
\newcommand{\FunctionTok}[1]{\textcolor[rgb]{0.02,0.16,0.49}{#1}}
\newcommand{\ImportTok}[1]{#1}
\newcommand{\InformationTok}[1]{\textcolor[rgb]{0.38,0.63,0.69}{\textbf{\textit{#1}}}}
\newcommand{\KeywordTok}[1]{\textcolor[rgb]{0.00,0.44,0.13}{\textbf{#1}}}
\newcommand{\NormalTok}[1]{#1}
\newcommand{\OperatorTok}[1]{\textcolor[rgb]{0.40,0.40,0.40}{#1}}
\newcommand{\OtherTok}[1]{\textcolor[rgb]{0.00,0.44,0.13}{#1}}
\newcommand{\PreprocessorTok}[1]{\textcolor[rgb]{0.74,0.48,0.00}{#1}}
\newcommand{\RegionMarkerTok}[1]{#1}
\newcommand{\SpecialCharTok}[1]{\textcolor[rgb]{0.25,0.44,0.63}{#1}}
\newcommand{\SpecialStringTok}[1]{\textcolor[rgb]{0.73,0.40,0.53}{#1}}
\newcommand{\StringTok}[1]{\textcolor[rgb]{0.25,0.44,0.63}{#1}}
\newcommand{\VariableTok}[1]{\textcolor[rgb]{0.10,0.09,0.49}{#1}}
\newcommand{\VerbatimStringTok}[1]{\textcolor[rgb]{0.25,0.44,0.63}{#1}}
\newcommand{\WarningTok}[1]{\textcolor[rgb]{0.38,0.63,0.69}{\textbf{\textit{#1}}}}
\usepackage{graphicx,grffile}
\makeatletter
\def\maxwidth{\ifdim\Gin@nat@width>\linewidth\linewidth\else\Gin@nat@width\fi}
\def\maxheight{\ifdim\Gin@nat@height>\textheight\textheight\else\Gin@nat@height\fi}
\makeatother
% Scale images if necessary, so that they will not overflow the page
% margins by default, and it is still possible to overwrite the defaults
% using explicit options in \includegraphics[width, height, ...]{}
\setkeys{Gin}{width=\maxwidth,height=\maxheight,keepaspectratio}
\setlength{\emergencystretch}{3em}  % prevent overfull lines
\providecommand{\tightlist}{%
  \setlength{\itemsep}{0pt}\setlength{\parskip}{0pt}}
\setcounter{secnumdepth}{0}
% Redefines (sub)paragraphs to behave more like sections
\ifx\paragraph\undefined\else
\let\oldparagraph\paragraph
\renewcommand{\paragraph}[1]{\oldparagraph{#1}\mbox{}}
\fi
\ifx\subparagraph\undefined\else
\let\oldsubparagraph\subparagraph
\renewcommand{\subparagraph}[1]{\oldsubparagraph{#1}\mbox{}}
\fi

% set default figure placement to htbp
\makeatletter
\def\fps@figure{htbp}
\makeatother


\date{}

\begin{document}

\hypertarget{projecte-asix-2k22}{%
\section{\texorpdfstring{\textbf{Projecte ASIX
2k22}}{Projecte ASIX 2k22}}\label{projecte-asix-2k22}}

\hypertarget{escola-del-treball}{%
\subsection{\texorpdfstring{\textbf{Escola Del
Treball}}{Escola Del Treball}}\label{escola-del-treball}}

\hypertarget{hisx-2021-2022}{%
\subsubsection{\texorpdfstring{\textbf{2HISX
2021-2022}}{2HISX 2021-2022}}\label{hisx-2021-2022}}

\hypertarget{aaron-andal-cristian-condolo}{%
\subsubsection{\texorpdfstring{\textbf{Aaron Andal \& Cristian
Condolo}}{Aaron Andal \& Cristian Condolo}}\label{aaron-andal-cristian-condolo}}

\hypertarget{cryptosec-careful-where-you-step-in}{%
\section{\texorpdfstring{\textbf{CryptoSEC}: ``\emph{Careful where you
step
in}''}{CryptoSEC: ``Careful where you step in''}}\label{cryptosec-careful-where-you-step-in}}

\includegraphics{./tex2pdf.-0a4f409d8d6fb9b0/785b57206d80bd3d5ef15f036481e103db8435a1.png}

\hypertarget{index}{%
\section{\texorpdfstring{\textbf{Index}}{Index}}\label{index}}

\begin{itemize}
\item
  \textbf{Que es el DNS?}:
  \protect\hyperlink{que-es-el-dns}{--\textgreater{} readME
  \textless{}--}
\item
  \textbf{Com funciona el DNS}:
  \protect\hyperlink{com-funciona-el-dns}{--\textgreater{} readME
  \textless{}--}
\item
  \textbf{Tipus de DNS ``4 servidors DNS implicats en la càrrega d'una
  pàgina web''}:
  \protect\hyperlink{tipus-de-dns-ux5cux25224-servidors-dns-implicats-en-la-cuxe0rrega-duna-puxe0gina-webux5cux2522}{--\textgreater{}
  readME \textless{}--}

  \begin{itemize}
  \item
    \textbf{Resolver de DNS Recursiu}:
    \protect\hyperlink{resolver-de-dns-recursiu}{--\textgreater{} readME
    \textless{}--}
  \item
    \textbf{Root Servers}:
    \protect\hyperlink{root-servers}{--\textgreater{} readME
    \textless{}--}
  \item
    \textbf{Servidor DNS - TLD}:
    \protect\hyperlink{servidor-dns---tld}{--\textgreater{} readME
    \textless{}--}
  \item
    \textbf{Servidor DNS Authoritative}:
    \protect\hyperlink{servidor-dns-authoritative}{--\textgreater{}
    readME \textless{}--}
  \item
    \textbf{Diferencia entre ``Authoritative DNS Server'' i ``Recursive
    DNS Resolver''}:
    \protect\hyperlink{diferencia-entre-ux5cux2522authoritative-dns-serverux5cux2522-i-ux5cux2522recursive-dns-resolverux5cux2522}{--\textgreater{}
    readME \textless{}--}
  \item
    \textbf{Recursive DNS Resolver}:
    \protect\hyperlink{recursive-dns-resolver}{--\textgreater{} readME
    \textless{}--}
  \item
    \textbf{Authoritative DNS Server}:
    \protect\hyperlink{authoritative-dns-server}{--\textgreater{} readME
    \textless{}--}
  \end{itemize}
\item
  \textbf{Procediment per fer un ``lookup'' de DNS}:
  \protect\hyperlink{procediment-per-fer-un-ux5cux2522lookupux5cux2522-de-dns}{--\textgreater{}
  readME \textless{}--}

  \begin{itemize}
  \tightlist
  \item
    \textbf{Els 8 passos d'un ``lookup'' de DNS}:
    \protect\hyperlink{els-8-passos-dun-ux5cux2522lookupux5cux2522-de-dns}{--\textgreater{}
    readME \textless{}--}
  \end{itemize}
\item
  \textbf{Què és un \emph{``resolver''} de DNS}:
  \protect\hyperlink{quuxe8-uxe9s-un-ux5cux2522resolverux5cux2522-de-dns}{--\textgreater{}
  readME \textless{}--}
\item
  \textbf{Tipus de consultes DNS}:
  \protect\hyperlink{tipus-de-consultes-dns}{--\textgreater{} readME
  \textless{}--}

  \begin{itemize}
  \tightlist
  \item
    \textbf{3 tipus de consultes DNS}:
    \protect\hyperlink{3-tipus-de-consultes-dns}{--\textgreater{} readME
    \textless{}--}
  \end{itemize}
\item
  \textbf{Que es el emmagatzematge en caché de DNS?}:
  \protect\hyperlink{que-es-el-emmagatzematge-en-cachuxe9-de-dns}{--\textgreater{}
  readME \textless{}--}
\item
  \textbf{Que es un registre DNS?}:
  \protect\hyperlink{que-es-un-registre-dns}{--\textgreater{} readME
  \textless{}--}
\item
  \textbf{Tipus de registres DNS}:
  \protect\hyperlink{tipus-de-registres-dns}{--\textgreater{} readME
  \textless{}--}
\item
  \textbf{Que es un DNS recursiu?}:
  \protect\hyperlink{que-es-un-dns-recursiu}{--\textgreater{} readME
  \textless{}--}
\item
  \textbf{Exemple resumit de DNS}:
  \protect\hyperlink{exemple-resumit-de-dns}{--\textgreater{} readME
  \textless{}--}

  \begin{itemize}
  \item
    \textbf{Diferencia entre recursió i iteració}:
    \protect\hyperlink{diferencia-entre-recursiuxf3-i-iteraciuxf3}{--\textgreater{}
    readME \textless{}--}

    \begin{itemize}
    \tightlist
    \item
      \textbf{Iteració - Recursió / Resum}:
      \protect\hyperlink{iteraciuxf3---recursiuxf3--resum}{--\textgreater{}
      readME \textless{}--}
    \end{itemize}
  \item
    \textbf{Advantatges del DNS recursiu}:
    \protect\hyperlink{advantatges-del-dns-recursiu}{--\textgreater{}
    readME \textless{}--}
  \item
    \textbf{Desaventatges del DNS recursiu}:
    \protect\hyperlink{desaventatges-del-dns-recursiu}{--\textgreater{}
    readME \textless{}--}
  \item
    \textbf{Servidors DNS recursius i atacs d'amplificació de DNS}:
    \protect\hyperlink{servidors-dns-recursius-i-atacs-damplificaciuxf3-de-dns}{--\textgreater{}
    readME \textless{}--}
  \item
    \textbf{Servidors DNS recursius i atacs d'enverinament de caché de
    DNS}:
    \protect\hyperlink{servidors-dns-recursius-i-atacs-denvergament-de-cachuxe9-de-dns}{--\textgreater{}
    readME \textless{}--}
  \end{itemize}
\item
  \textbf{Configuració DNS CryptoSEC}:
  \protect\hyperlink{configuraciuxf3-dns-cryptosec}{--\textgreater{}
  readME \textless{}--}

  \begin{itemize}
  \item
    \textbf{Instal·lació}:
    \protect\hyperlink{installaciuxf3}{--\textgreater{} readME
    \textless{}--}
  \item
    \textbf{El servidor DNS Autoritatiu}:
    \protect\hyperlink{el-servidor-dns-autoritatiu}{--\textgreater{}
    readME \textless{}--}

    \begin{itemize}
    \item
      \textbf{Arxiu de d'opcions de les zones}:
      \protect\hyperlink{arxiu-de-dopcions-de-les-zones}{--\textgreater{}
      readME \textless{}--}
    \item
      \textbf{Arxiu de dades per a una zona directa
      "\emph{cryptosec.net}}:
      \protect\hyperlink{arxiu-de-dades-per-a-una-zona-directa-ux5cux2522cryptosecnetux5cux2522}{--\textgreater{}
      readME \textless{}--}
    \end{itemize}
  \item
    \textbf{El servidor DNS Forwarder}:
    \protect\hyperlink{el-servidor-dns-forwarder}{--\textgreater{}
    readME \textless{}--}

    \begin{itemize}
    \item
      \textbf{Arxiu de d'opcions de les zones}:
      \protect\hyperlink{arxiu-de-dopcions-de-les-zones-1}{--\textgreater{}
      readME \textless{}--}
    \item
      \textbf{Arxiu de dades per especificar el forwarding a la zona
      "\emph{cryptosec.net}}:
      \protect\hyperlink{arxiu-de-dades-per-especificar-el-forwarding-a-la-zona-ux5cux2522cryptosecnetux5cux2522}{--\textgreater{}
      readME \textless{}--}
    \end{itemize}
  \item
    \textbf{Comandes de verificació}:
    \protect\hyperlink{comandes-de-verificaciuxf3}{--\textgreater{}
    readME \textless{}--}
  \item
    \textbf{El client DNS}:
    \protect\hyperlink{el-client-dns}{--\textgreater{} readME
    \textless{}--}
  \item
    \textbf{Resolució de noms al client}:
    \protect\hyperlink{resoluciuxf3-de-noms-al-client}{--\textgreater{}
    readME \textless{}--}

    \begin{itemize}
    \item
      \textbf{Exemple de /etc/hosts}:
      \protect\hyperlink{exemple-de-etchosts}{--\textgreater{} readME
      \textless{}--}
    \item
      \textbf{Exemple de /etc/resolv.conf}:
      \protect\hyperlink{exemple-de-etcresolvconf}{--\textgreater{}
      readME \textless{}--}
    \end{itemize}
  \end{itemize}
\item
  \textbf{El servei systemd-resolved i la comanda resolvectl}:
  \protect\hyperlink{el-servei-systemd-resolved-i-la-comanda-resolvectl}{--\textgreater{}
  readME \textless{}--}
\item
  \textbf{Com donar suport a consultes de DNS ràpides i segures}:
  \protect\hyperlink{com-donar-suport-a-consultes-de-dns-ruxe0pides-i-segures}{--\textgreater{}
  readME \textless{}--}
\item
  \textbf{Glossari de termes de les configuracions de BIND9}:
  \protect\hyperlink{glossari-de-termes-de-les-configuracions-de-bind9}{--\textgreater{}
  readME \textless{}--}
\item
  \textbf{Glossari de termes de tipus de servidors de BIND9}:
  \protect\hyperlink{glossari-de-termes-de-tipus-de-servidors-de-bind9}{--\textgreater{}
  readME \textless{}--}
\item
  \textbf{Glossari de termes seongs cada camp del SOA amb la seva funció
  (Bind9)}:
  \protect\hyperlink{glossari-de-termes-seongs-cada-camp-del-soa-amb-la-seva-funciuxf3-bind9}{--\textgreater{}
  readME \textless{}--}
\item
  \textbf{Exemples BIND9 (Configuracions)}:
  \protect\hyperlink{exemples-bind9-configuracions}{--\textgreater{}
  readME \textless{}--}

  \begin{itemize}
  \item
    \textbf{GLUE RECORD}:
    \protect\hyperlink{glue-record}{--\textgreater{} readME
    \textless{}--}
  \item
    \textbf{\$GENERATE}: \protect\hyperlink{generate}{--\textgreater{}
    readME \textless{}--}
  \item
    \textbf{Resolució inversa}:
    \protect\hyperlink{resoluciuxf3-inversa}{--\textgreater{} readME
    \textless{}--}
  \end{itemize}
\item
  \textbf{Bibliografia}:
  \protect\hyperlink{bibliografia}{--\textgreater{} readME
  \textless{}--}
\end{itemize}

\hypertarget{que-es-el-dns}{%
\section{\texorpdfstring{\textbf{Que es el
DNS?}}{Que es el DNS?}}\label{que-es-el-dns}}

El sistema de noms de \textbf{domini (DNS)} és l'agenda telefònica
d'Internet. Permet associar noms de domini amb direccions IP per
facilitar en gran mesura l'accés als hosts de la xarxa.

Els humans accedeixen a la informació en línia mitjançant \textbf{noms
de domini}, com ara \textbf{nytimes.com} o \textbf{espn.com}.

Els navegadors web interactuen mitjançant adreces de \textbf{Protocol
d'Internet (IP)}.

DNS \emph{tradueix} els \textbf{noms de domini} a \textbf{adreces IP}
perquè els navegadors puguin \emph{carregar recursos d'Internet}.

Cada dispositiu connectat a \textbf{Internet} té una \textbf{adreça IP
única} que altres màquines utilitzen per trobar el dispositiu.

Els servidors DNS eliminen la necessitat que els h\_\_umans
memoritzin\_\_ adreces IP com ara \textbf{192.168.1.1} (en IPv4) o
adreces IP alfanumèriques més complexes, com ara
2400:cb00:2048:1::c629:d7a2 (en IPv6).

\includegraphics{./tex2pdf.-0a4f409d8d6fb9b0/778e9507c2aa821fec43aad3592f3c228e52a60e.svg}

\hypertarget{com-funciona-el-dns}{%
\section{\texorpdfstring{\textbf{Com funciona el
DNS}}{Com funciona el DNS}}\label{com-funciona-el-dns}}

El procés de resolució de DNS implica convertir un nom
d'\textbf{amfitrió} (com ara www.example.com) en una adreça IP
compatible amb l'ordinador (com ara 192.168.1.1). Es dóna una adreça IP
a cada dispositiu a \textbf{Internet}, i aquesta adreça és necessària
per trobar el dispositiu d'Internet adequat, com si s'utilitza una
\textbf{adreça} de carrer per trobar una casa determinada.

Quan un usuari vol \textbf{carregar} una \textbf{pàgina web}, s'ha de
produir una traducció entre el que un usuari escriu al seu navegador web
(\textbf{example.com}) i l'adreça adaptada a la màquina necessària per
localitzar la pàgina web \textbf{example.com}.

Per entendre el procés darrere de la \emph{resolució DNS}, és important
conèixer els diferents components de maquinari entre els quals ha de
passar una consulta DNS. Per al navegador web, la cerca de DNS es
produeix ``darrere de l'escenari'' i no requereix cap interacció de
l'ordinador de l'usuari a part de la sol·licitud inicial.

\includegraphics{./tex2pdf.-0a4f409d8d6fb9b0/8d313c7da075c3c7303aaef32e89b5d0b7885e7c.png}

\hypertarget{tipus-de-dns-4-servidors-dns-implicats-en-la-cuxe0rrega-duna-puxe0gina-web}{%
\section{\texorpdfstring{\textbf{Tipus de DNS ``4 servidors DNS
implicats en la càrrega d'una pàgina
web'':}}{Tipus de DNS ``4 servidors DNS implicats en la càrrega d'una pàgina web'':}}\label{tipus-de-dns-4-servidors-dns-implicats-en-la-cuxe0rrega-duna-puxe0gina-web}}

Tots els \textbf{servidors DNS} es divideixen en una d'aquestes quatre
categories: \textbf{Resolvers recursius}, \textbf{Root Servers} ,
\textbf{Servidor de noms TLD} i \textbf{State Of Authority}.

En una cerca DNS típica (quan no hi ha \textbf{memòria cau} en joc),
aquests quatre servidors DNS treballen junts en harmonia per completar
la tasca de lliurar l' adreça IP d'un domini especificat al client (el
client sol ser un solucionador de talons, un senzill resolutor integrat
en un sistema operatiu).

\hypertarget{resolver-de-dns-recursiu}{%
\subsection{\texorpdfstring{\textbf{Resolver de DNS
Recursiu}}{Resolver de DNS Recursiu}}\label{resolver-de-dns-recursiu}}

\begin{enumerate}
\def\labelenumi{\arabic{enumi}.}
\tightlist
\item
  \textbf{DNS Recursor (Servidor DNS Recursiu)} : És com un
  \textbf{bibliotecari} a la qual se li demana que busqui un llibre
  determinar a la biblioteca. El \textbf{recurs DNS} és un
  \textbf{servidor} dissenyat per rebre consultes de les màquines client
  mitjançant aplicacions com ara navegadors \textbf{web}. Normalment, el
  recurs és responsable de fer \textbf{peticions} addicionals per
  satisfer la \textbf{consulta DNS del client}.
\end{enumerate}

És la \emph{primera parada} d'una \textbf{consulta DNS}.

El \emph{resolver} recursiu actua com a intermediari entre un client i
un servidor de noms DNS.

Després de rebre una consulta DNS d'un client web, un \emph{resolver}
recursiu respondrà amb dades emmagatzemades a la \textbf{memoria CAU} o
enviarà una sol·licitud a un servidor de \textbf{noms arrel}
\textbf{(Root Servers)}, seguida d'una altra sol·licitud a un servidor
de noms TLD i després d'una úlima sol·licitud a un servidor de noms
autoritzat.

Després de rebre una resposta del servidor de noms autoritzat que conté
l'adreça IP sol·licitada, el resolutor recursiu envia una resposta al
client.

Durant aquest procés, el solucionador recursiu guardarà a la
\textbf{memòria cau} la informació rebuda dels servidors de noms
autoritzats.

Quan un client nou sol·liciti l'adreça IP d'un nom de domini que ha
estat sol·licitat recentment per un altre client, el \emph{resolver} pot
eludir el procés de comunicació amb els servidors de noms i només
lliurar al client el registre sol·licitat de la seva memòria cau.

La majoria dels usuaris d'Internet utilitzen un solucionador recursiu
proporcionat pel seu ISP, però hi ha altres opcions disponibles; per
exemple , l'1.1.1.1 de Cloudflare .

\includegraphics{./tex2pdf.-0a4f409d8d6fb9b0/6e363b041c396b38ad59df1bce0c347ebadbfcfc.png}

\hypertarget{root-servers}{%
\subsection{\texorpdfstring{\textbf{Root
Servers}}{Root Servers}}\label{root-servers}}

\begin{itemize}
\tightlist
\item
  \textbf{Root Servers} : El \textbf{servidor arrel} és el primer pas
  per traduir (resolució) noms d'amfitrió llegibles per humans a
  \textbf{adreces IP}. Es pot pensar com un \textbf{índex} en una
  \textbf{biblioteca} que apunta a \textbf{diferents bastidors} de
  llibres; normalment serveix com a referència a altres ubicacions més
  específiques.
\end{itemize}

Hi han 13 servidors de noms d'arrel DNS, són coneguts per tots els
\emph{resolvers recursius} i són la primera parada en la recera de
registres DNS d'un \emph{resolver recursiu}.

Un servidor arrel accepta la consulta d'un resolutor recursiu que inclou
un nom de domini, i el servidor de noms arrel respon dirigint el
resolutor recursiu a un \textbf{servidor de noms TLD}, en funció de
l'extensió d'aquest domini \textbf{(.com, .net, .org, etc.)}.

\includegraphics{./tex2pdf.-0a4f409d8d6fb9b0/6e363b041c396b38ad59df1bce0c347ebadbfcfc.png}

\hypertarget{servidor-dns---tld}{%
\subsection{\texorpdfstring{\textbf{Servidor DNS -
TLD}}{Servidor DNS - TLD}}\label{servidor-dns---tld}}

\begin{itemize}
\tightlist
\item
  \textbf{Servidor DNS - TLD} : El servidor de domini de \textbf{primer
  nivell} ( TLD = Top Layer Domain ) es pot considerar com un \emph{lloc
  específic} de llibres d'una biblioteca. Aquest servidor de noms és el
  següent pas en la cerca d'una adreça IP específica i allotja l'última
  part d'un nom d'amfitrió (a example.com, el servidor TLD és ``com'').
\end{itemize}

Un \textbf{servidor de noms TLD} manté la informació de tots els noms de
domini que comparteixen una extensió de domini comuna, com ara .com,
.net o qualsevol que vingui després de l'últim punt d'una URL.

Per exemple, un servidor de noms TLD \textbf{.com} conté informació per
a cada lloc web que acabi en \textbf{``.com''}.

Si un usuari estava cercant \textbf{google.com}, després de rebre una
resposta d'un servidor de noms arrel, el solucionador recursiu enviaria
una consulta a un servidor de noms TLD .com, que respondria apuntant al
servidor de noms autoritzat (vegeu més avall) per a aquest domini.

\begin{itemize}
\item
  Dominis genèrics de \textbf{primer nivell}: són dominis que no són
  específics d'un país, alguns dels TLD genèrics més coneguts inclouen
  \textbf{.com, .org, .net, .edu i .gov.}
\item
  Dominis de \textbf{nivell superior} de codi de país: inclouen tots els
  dominis específics d'un país o estat. Alguns exemples inclouen
  \textbf{.uk, .us, .ru i .jp}.
\end{itemize}

\includegraphics{./tex2pdf.-0a4f409d8d6fb9b0/d849fbdf3da4f18477429f29a2025aa53c06a8b8.png}

\hypertarget{servidor-dns-authoritative}{%
\subsection{\texorpdfstring{\textbf{Servidor DNS
Authoritative}}{Servidor DNS Authoritative}}\label{servidor-dns-authoritative}}

\begin{itemize}
\tightlist
\item
  \textbf{Servidor de noms autoritzat (Authoritative DNS Server)} : Es
  pot interpretar com un diccionari en una \textbf{prestatgeria de
  llibres}, on es pot consultar la \textbf{definició} d'un \textbf{nom
  específic}. Aquest servidor de noms autoritzat és \textbf{l'última
  parada} de la consulta del servidor de noms. Si el servidor de noms
  autoritzat té accés al registre sol·licitat, retornarà l'adreça IP del
  nom d'amfitrió sol·licitat al recurs DNS \textbf{(el bibliotecari)}
  que va fer la sol·licitud inicial.
\end{itemize}

Quan un \emph{resolver recursiu} rep una resposta d'un servidor de noms
TLD, aquesta resposta es dirigirà directament a un servidor DNS
autoritatiu \emph{(Authoritative DNS Server)}.

El servidor de noms autoritatiu conté la informació específica del nom
de domini a la qual serveix.

Pot proporcionar una \textbf{solució recursiva} amb l'adreça IP d'aquest
servidor que es troba al registra \textbf{DNS A} o si té un alias
registre \textbf{CNAME}, que proporcionarà al \emph{resolver recursiu}
un domini d'àlies.

\includegraphics{./tex2pdf.-0a4f409d8d6fb9b0/a1ae1dd37c864176bc3c8c5ad7eca3b50da996bc.png}

\hypertarget{diferencia-entre-authoritative-dns-server-i-recursive-dns-resolver}{%
\section{\texorpdfstring{\textbf{Diferencia entre ``Authoritative DNS
Server'' i ``Recursive DNS
Resolver''}}{Diferencia entre ``Authoritative DNS Server'' i ``Recursive DNS Resolver''}}\label{diferencia-entre-authoritative-dns-server-i-recursive-dns-resolver}}

Els dos conceptes es refereixen a servidors (O grups de servidors) que
estàn \emph{``integrals''} a la infraestructura DNS, però cadascun
realitza un paper diferent. Es troba en diferents ubicacions dins del
trajecte d'una consulta de DNS. Una manera d'entendre la diferència és
que el \emph{``Recursive Resolver''} és a l'inici de la consulta DNS i
el \emph{``Athoritative Nameserver''} (servidor de noms autoritatiu) al
final.

\hypertarget{recursive-dns-resolver}{%
\subsection{\texorpdfstring{\textbf{Recursive DNS
Resolver}}{Recursive DNS Resolver}}\label{recursive-dns-resolver}}

El \emph{``Recursive DNS Resolver''} és el servidor que respon a una
solicitud recursiva del client i dedica temps a detectar el
\textbf{registre DNS}.

Ho fa mitjançant una \textbf{sèrie de sol·licituds} fins que arriba al
servidor de noms DNS autoritatiu \emph{(Authoritative DNS Server)} per
al registre DNS sol·licitat (o es torna inactiu o torna un error si no
es troba cap registre).

Amb altres paraules, el client que fa una petició per anar a
\emph{www.exemple.com}, el \emph{``Recursive DNS Resolver''} respon la
petició i va preguntant a altres \emph{servidors} quina IP es la
\emph{www.exemple.com} fins que arriba al servidor DNS autoriratiu
\emph{(Authoritative DNS Server)} que conté la zona
\emph{www.exemple.com} en els seus registres.

Afortunadament, els \emph{``Recursive DNS Resolver''} \textbf{no} sempre
han de fer diverses sol·licituds per inspeccionar els registres
necessaris per respondre a un client.

L' emmagatzematge en \textbf{memòria cau} és un procés d'agilització de
procés en la \emph{busca} del registre DNS que ajuda a
\textbf{saltar-se} les sol·licituds \emph{necessàries} servint abans el
registre del recurs sol·licitat a la cerca DNS.

\includegraphics{./tex2pdf.-0a4f409d8d6fb9b0/7f70b1b7a782c350c2be9bdab25878acd4b57603.png}

\hypertarget{authoritative-dns-server}{%
\subsection{\texorpdfstring{\textbf{Authoritative DNS
Server}}{Authoritative DNS Server}}\label{authoritative-dns-server}}

Un servidor DNS autoritatiu \emph{(Authoritative DNS Server)} és un
servidor que allotja realment registres de recursos DNS i n'és
responsable.

Aquest és el \textbf{servidor al final} de la cadena de cerca de DNS que
respondrà amb el registre DNS del recurs consultat, permetent finalment
que el \textbf{navegador web} faci la sol·licitud per arribar a
\textbf{l'adreça IP} necessària per accedir a un \textbf{lloc web} o
altres \textbf{recursos web}.

Un servidor DNS autoritatiu \emph{(Authoritative DNS Server)} pot oferir
sol·licituds a partir de les seves pròpies dades sense necessitat de
consultar altres recursos \emph{(recursive)}, ja que és la font final de
veritat per a \textbf{certs registres DNS}.

\includegraphics{./tex2pdf.-0a4f409d8d6fb9b0/5b3e45c83580b88d534d409b39c8d05bca47027a.png}

Convé indicar que en els casos de \emph{consultes relatives} a
\textbf{subdominis}, com ara \emph{foo.exemple.com} o
\emph{blog.cloudflare.com} , s'afegirà un servidor de noms addicional a
la seqüència després del servidor de noms autoritatiu, que és el
responsable d'emmagatzemar el registre \textbf{CNAME} del subdomini.

\includegraphics{./tex2pdf.-0a4f409d8d6fb9b0/5c5bfcbebb966de717b62f6fbf3d7fd7a1b0bb20.png}

Hi ha una diferencia fundamental entre molts serveis de DNS i el que
ofereix Cloudfare per exemple. Hi han diferents \emph{``Recursive DNS
Resolver''} com Google DNS, OpenDNS o proveïdors com Comcast mantenen
instal·lacions de centre de dades de \emph{``Recursive DNS Resolver''}.

Aquests \emph{``Resolvers''} permeten consultes rapides i senzilles
mitjaçant clusters optimitzats de sistemes informàtics optimitzats per a
DNS. Però son bàsicament diferents servidors de noms allotjats en
servidors com per exemple Cloudflare.

\hypertarget{procediment-per-fer-un-lookup-de-dns}{%
\section{\texorpdfstring{\textbf{Procediment per fer un ``lookup'' de
DNS}}{Procediment per fer un ``lookup'' de DNS}}\label{procediment-per-fer-un-lookup-de-dns}}

En la majoria de situacions, DNS fa referència a un \textbf{nom de
domini} que s'està traduint a l'adreça IP.

Sovint, la informació de cerca de DNS s'emmagatzemarà a la memòria
\textbf{cau local} dins del servidor que realitzi la \textbf{consulta} o
en \textbf{remot} a la \emph{infraestructura} de DNS.

Generalment, hi ha 8 passos en una cerca DNS.

Quan la informació de DNS s'emmagatzema en memòria cau, s'ometen els
passos del procés de cerca DNS, cosa que ho fa més ràpid. L'exemple
descriu els 8 passos necessaris quan no s'ha emmagatzemat res a la
memòria cau.

\hypertarget{els-8-passos-dun-lookup-de-dns}{%
\subsection{\texorpdfstring{\textbf{Els 8 passos d'un ``lookup'' de
DNS}}{Els 8 passos d'un ``lookup'' de DNS}}\label{els-8-passos-dun-lookup-de-dns}}

\begin{enumerate}
\def\labelenumi{\arabic{enumi}.}
\item
  Un usuari escriu ``exemple.com'' en un navegador web i la consulta
  recorre Internet i és rebuda per un \emph{``resolutor recursiu de DNS
  (Recursive DNS Resolver)''}.
\item
  El \emph{resolver} consulta a continuació un servidor de \emph{noms
  d'arrel de DNS} \emph{(Root Servers) (.)} = Internet.
\item
  El servidor arrel \emph{(Root Servers)} respon a continuació al
  \emph{resolver} amb l'adreça d'un servidor de \emph{DNS de domini} de
  \textbf{primer nivell} (TLD = Top Level Domain) (p.ex. .com o .net),
  que emmagatzema la informació per als vostres dominis. En cercar
  ``\emph{exemple.com}'', la nostra sol·licitud es dirigeix al TLD
  \textbf{.com}.
\item
  El \emph{resolver} farà a continuació una sol·licitud al domini de
  \textbf{primer nivell} .\_\_com\_\_.
\item
  El \textbf{servidor TLD} respondrà a continuació amb l'adreça IP del
  servidor de noms del domini autoriratiu \emph{(Authoritative DNS
  Server)}: \textbf{exemple.com}.
\item
  Finalment, el \emph{resolver recursiu} envia una consulta al
  \emph{servidor de noms del domini autoritatiu} \textbf{(Authoritative
  DNS Server)}.
\item
  Per exemple, l'adreça IP es retornarà al \emph{resolver} desde del
  \emph{servidor de noms}.
\item
  El \emph{resolver de DNS} respondrà a continuació al \emph{navegador
  web} amb \textbf{l'adreça IP del domini} sol·licitat inicialment.
\end{enumerate}

Un cop els 8 passos de la cerca del DNS han tornat l'adreça IP per
exemple.com, el navegador podrà fer la sol·licitud per a la \emph{pàgina
web}:

\begin{enumerate}
\def\labelenumi{\arabic{enumi}.}
\setcounter{enumi}{8}
\item
  El navegador farà una sol·licitud \textbf{d'HTTP} a l'adreça IP.
\item
  El servidor en aquesta adreça IP torna la \textbf{pàgina web} perquè
  es processi al navegador (pas 10).
\end{enumerate}

\includegraphics{./tex2pdf.-0a4f409d8d6fb9b0/da9d0e1fe919108601650ca8681f91fabbe42363.png}

\hypertarget{quuxe8-uxe9s-un-resolver-de-dns}{%
\section{\texorpdfstring{\textbf{Què és un \emph{``resolver''} de
DNS}}{Què és un ``resolver'' de DNS}}\label{quuxe8-uxe9s-un-resolver-de-dns}}

El \textbf{resolver de DNS} és la \textbf{primera parada} de la recerca
de DNS i s'encarrega de tractar amb el \textbf{client} que va fer la
\textbf{sol·licitud inicial}.

El solucionador inicia la seqüència de consultes que porten en última
instància que l'URL es tradueixi a l'adreça IP necessària.

Nota: una cerca de DNS no emmagatzemada a la memòria cau inclourà
consultes recursives i iteratives.

És important diferencia entre una \emph{consulta} de \emph{DNS recursiu}
i un \emph{resolver} de \emph{DNS recursiu}.

La \textbf{consulta} fa referència a la sol·licitud feta a un
solucionador de DNS que requereix la resolució de la consulta.

Un \textbf{resolver} de DNS recursiu és el servidor que accepta una
\textbf{solució recursiva} i procesa la resposta fent les
\textbf{sol·licituds necessàries}.

\includegraphics{./tex2pdf.-0a4f409d8d6fb9b0/861a2b93b8a76d0cd7d5659cbaca80ca9711d655.png}

\hypertarget{tipus-de-consultes-dns}{%
\section{\texorpdfstring{\textbf{Tipus de consultes
DNS}}{Tipus de consultes DNS}}\label{tipus-de-consultes-dns}}

En una busca DNS habitual es produeixen 3 tipus de consultes.

En utilitzar una combinació d'aquestes consultes, un procés optimitzat
per la resolució de DNS es pot comportar una reducció de ``salts''. En
una situació ideal, les dades de registra emmagatzemades a la memòria
CAU estaran disponibles, la qual cosa permetrà que un servidor de noms
DNS torni a una consulta no \emph{recursiva}.

\hypertarget{tipus-de-consultes-dns-1}{%
\subsection{\texorpdfstring{\textbf{3 tipus de consultes
DNS:}}{3 tipus de consultes DNS:}}\label{tipus-de-consultes-dns-1}}

\begin{enumerate}
\def\labelenumi{\arabic{enumi}.}
\item
  \textbf{Consulta recursiva}: En una consulta recursiva, un client DNS
  requereix que un servidor DNS (generalment un \emph{resolver} de
  \textbf{DNS recursiu}) respongui al \textbf{client} amb el registre
  del recurs sol·licitat o un missatge d'error si el solucionador no pot
  trobar el registre.
\item
  \textbf{Consulta iterativa}: En aquesta situació, el \emph{client DNS}
  permetrà que un \textbf{servidor DNS} retorni la \textbf{millor
  resposta} possible. Si el servidor DNS consultat no té el \textbf{nom}
  que ha demanat el client en la s eva consulta, el servidor DNS
  retornarà una referencia a un servidor DNS autoritatiu. El
  \textbf{client DNS} farà a continuació una consulta a \textbf{l'adreça
  de referència}. Aquest procés continua amb servidors DNS addicionals
  que segueixen a la cadena de consulta fins que es produeixi un error o
  se superi el temps despera.
\item
  \textbf{Consulta no recursiva}: Generalment es produeix quan un
  \textbf{client solucionador de DNS} consulta un \textbf{servidor DNS}
  per un registre al qual té \textbf{accés} perquè o bé \textbf{és
  autoritatiu} per al \textbf{registre} o el registre \textbf{existeix}
  dins de la seva memòria cau. Generalment, el servidor DNS
  emmagatzemarà a la memòria cau registres DNS per prevenir el consum
  d'amplada de banda addicional i la càrrega als servidors que
  precedeixen a la cadena.
\end{enumerate}

\hypertarget{que-es-el-emmagatzematge-en-cachuxe9-de-dns}{%
\section{\texorpdfstring{\textbf{Que es el emmagatzematge en caché de
DNS?}}{Que es el emmagatzematge en caché de DNS?}}\label{que-es-el-emmagatzematge-en-cachuxe9-de-dns}}

L'objectiu de l'emmagatzematge a la memòria cau és guardar dades en una
ubicació temporalment per aconseguir millores en el rendiment i
fiabilitat en les sol·licituds de dades.

L'emmagatzematge en memòria cau de DNS guarda dades més a prop del
client sol·licitant perquè la consulta DNS es pugui resoldre abans i les
consultes addicionals que segueixen a la cadena de cerca DNS es puguin
evitar, millorant així els temps de càrrega i reduint el consum
d'amplada de banda/CPU.

Les dades de DNS es poden emmagatzemar en memòria cau en diverses
ubicacions. Cadascuna guardarà els registres DNS durant una quantitat de
temps establerta, determinada pel temps de vida (TTL) .

\hypertarget{que-es-un-registre-dns}{%
\section{\texorpdfstring{\textbf{Que es un registre
DNS?}}{Que es un registre DNS?}}\label{que-es-un-registre-dns}}

Els registres DNS o també conegut con \emph{arxius de zona} son
instruccions radicades en servidors DNS autoritatius que proporcionen
informacó sobre un domini, com l'adreça IP associada amb aquest i com
gestionar sol·licituds dirigides a aquest domini.

Aquests registres consisteixen en una sèrie de fitxers de text escrits
en el que es coneix com a sintaxi DNS. La sintaxi DNS és simplement una
cadena de caràcters utilitzats com a ordres que diuen al servidor DNS
què fer.

Exemple: \textbf{db.cryptosec.net} que es troba a \textbf{/etc/bind/}

Tots els registres DNS tenen també un " TTL ``, que vol
dir''time-to-live" i indica amb quina freqüència el servidor DNS
actualitzarà aquest registre.

\hypertarget{tipus-de-registres-dns}{%
\section{\texorpdfstring{\textbf{Tipus de registres
DNS}}{Tipus de registres DNS}}\label{tipus-de-registres-dns}}

\emph{MÉS COMUNS}

\begin{itemize}
\item
  A: Conté l'adreça IP d'un domini.
\item
  AAAA: Lo mateix que l'anterior pero per a Ipv6
\item
  CNAME: Reenvia un domini o subdominis, es un alias, no proporciona una
  adreça IP.
\item
  MX: Es dirigeix a un servidor de correu electrònic.
\item
  TXT: Permet que un administrador pugui emmagatzemar notes de text al
  registre. Aquests registres se solen utilitzar per a la seguretat del
  correu electrònic.
\item
  NS: Emmagatzema el servidor de noms per a una entrada DNS.
  \href{https://www.cloudflare.com/es-es/learning/dns/dns-records/dns-soa-record/}{Més
  info}
\item
  SOA: State of Authority. Emmagatzema la informació de l'administrador
  sobre un domini o zona.
  \href{https://www.cloudflare.com/es-es/learning/dns/dns-records/dns-ns-record/}{Més
  info}
\item
  SRV: Especifica un port per a serveis específics
\item
  PTR: Proporciona un nom de domini a cerques inverses. Resolució
  inversa.
\end{itemize}

\emph{MÉNYS COMUNS}

\begin{itemize}
\item
  SSHFP: Aquest registre emmagatzema les ``empremtes digitals de la clau
  pública SSH''; SSH fa referència a ``Secure Shell'' i és un protocol
  de xarxa xifrat que permet la comunicació segura a una xarxa insegura.
\item
  RP: Aquest és el registre de la ``persona responsable'' i emmagatzema
  l'adreça de correu electrònic de la persona responsable del domini.
\item
  DCHID : el ``identificador DHCP'' emmagatzema informació per al
  protocol de configuració dinàmica de host (DHCP), un protocol de xarxa
  estandarditzat utilitzat a les xarxes IP.
\end{itemize}

\textbf{DNSSEC}

\begin{itemize}
\item
  CAA: És el registre d'``autorització d'autoritat de certificació'';
  permet que els propietaris d'un domini especifiquin quines autoritats
  de certificació poden emetre certificats per a aquest domini. Si no
  existeix cap registre CAA, aleshores qualsevol podrà emetre un
  certificat per a aquest domini. Aquests registres també els hereten
  els subdominis.
\item
  DNSKEY: El ' Registre de Clau DNS ' conté una clau pública que es fa
  servir per verificar les signatures de l' Extensió de seguretat del
  sistema de noms de domini (DNSSEC) .
\item
  CDNSKEY: És una còpia fill del registre DNSKEY destinada a
  transferir-se a un pare.
\item
  CERT: El ``registre de certificats'' emmagatzema certificats de claus
  públiques.
\item
  NSEC: El ``següent registre segur'' és part del DNSSEC i s'usa per
  demostrar que un registre de recursos DNS sol·licitat no existeix.
\item
  RRSIG : el ``registre de recursos de signatura'' emmagatzema
  signatures digitals utilitzades per autenticar registres de
  conformitat amb el DNSSEC.
\end{itemize}

\hypertarget{que-es-un-dns-recursiu}{%
\section{\texorpdfstring{\textbf{Que es un DNS
recursiu?}}{Que es un DNS recursiu?}}\label{que-es-un-dns-recursiu}}

Una cerca de DNS recursiu és quan un servidor DNS es comunica amb altres
servidors DNS per ``trobar'' una direcció IP i retornarla al client.
Això es diferencia d'una consulta de DNS iterativa, en la que el client
es comunica directament amb cada servidor DNS implicat en la cerca.

\hypertarget{exemple-resumit-de-dns}{%
\section{\texorpdfstring{\textbf{Exemple resumit de
DNS}}{Exemple resumit de DNS}}\label{exemple-resumit-de-dns}}

\begin{enumerate}
\def\labelenumi{\arabic{enumi}.}
\item
  Un usuari escriu un nom de domini: \textbf{``cryptosec.net''} en el
  seu navegador, s'activa una \textbf{cerca de DNS}.
\item
  Una serie d'ordinadors en \textbf{remot coneguts} com
  \textbf{servidors DNS} troben la \textbf{direcció IP} d'aquell
  \textbf{domini} i la retornen a l'ordinador de l'usuari per a que
  pugui accedir al lloc \textbf{web correcte}.
\item
  Diferents tipus de \textbf{servidors DNS} han de treballar
  conjuntament per poder completar aquesta cerca de DNS.
\item
  Actuén:

  \begin{itemize}
  \item
    Un solucionador o \textbf{resolver DNS}.
  \item
    Un servidor \textbf{Root Server} (1.1.1.1 o 8.8.8.8 per exemple).
  \item
    Un \textbf{servidor TLD} de DNS (de primer nivell, exemple: .net,
    .com\ldots{}).
  \item
    Un servidor de noms \textbf{autoritatiu} de DNS. Conté el
    \textbf{registre DNS}.
  \end{itemize}
\item
  Pot haber un cas de \textbf{``caching''} que algun d'aquests servidors
  pot haber emmagatzemat la respostas de la consulta durant una cerca
  anterior, llavors el client, en lloc de recorrer i esperar molt,
  tindrà un temps de resposta menor.
\end{enumerate}

\hypertarget{diferencia-entre-recursiuxf3-i-iteraciuxf3}{%
\subsection{\texorpdfstring{\textbf{Diferencia entre recursió i
iteració}}{Diferencia entre recursió i iteració}}\label{diferencia-entre-recursiuxf3-i-iteraciuxf3}}

La \textbf{recursió} i la \textbf{iteració} son termes informàtics que
descriuen dos mètodes diferents per resoldre un problema.

En la \textbf{recursió}, un programa es truca a \textbf{si mateix} fins
a que \textbf{cumpleixi} una \textbf{condició}. Mentes que la iteració
es repeteix un conjunt d'instruccions fins que es cumpleixi la condició.

Exemple: Jim ha perdut les seves claus de casa i les està buscant-la
d'una forma sistemàtica.

\begin{verbatim}
Una solució recursiva seria que Jim no pararia de buscar les claus. Jim començarà a buscar, i si no el troba, tornarà al punt de partida per tornar a buscar-les.

Una solució iterativa seria que Jim faci una cerca en una habitació durant 5 minuts i després tornaria a l'origen de les instruccions i buscaria en una altra habitació durant 5 minuts, repetiria aquest procés fins a ttrobar les claus o fins que hagi acabat la llista d'habiacions.
\end{verbatim}

En un servidor DNS que faci la \textbf{recursió} segurià consultant a
altres servidors DNS fins que obtingui la direcció IP a la qual pugui
retornar al client.

En un servidor DNS que faci una consulta \textbf{iterativa}, cada
consulta DNS respon \textbf{directament al client} amb una direcció per
a que un altre servidor DNS pregunti, i el client \textbf{seguirà
preguntant} a altres servidors DNS fins que algú d'ells responda amb la
IP i el domini correcte.

\hypertarget{iteraciuxf3---recursiuxf3-resum}{%
\subsubsection{\texorpdfstring{\textbf{Iteració - Recursió /
Resum:}}{Iteració - Recursió / Resum:}}\label{iteraciuxf3---recursiuxf3-resum}}

El client delega una consulta a un DNS recursiu:

\textbf{Recursiva}: ``Necessito la direcció IP d'aquest domini, per
favor, trobala i no tornis a trucar-me fins que la trobis''.

El client li diu al solucionador o \emph{``resolver''} de DNS:

\textbf{Iterativa}: ``Necessito la direcció IP d'aquest domini. Per
favor, dona'm la direcció següent del servidor DNS en el procés de la
recerca per a que jo mateix la pugui trobar''.

\hypertarget{advantatges-del-dns-recursiu}{%
\subsection{\texorpdfstring{\textbf{Advantatges del DNS
recursiu}}{Advantatges del DNS recursiu}}\label{advantatges-del-dns-recursiu}}

Les consultes DNS recursives solen resoldre's més ràpid que les
consultes iteratives. Això es degut al emmagatzematge \textbf{cache}. Un
servidor DNs recursiu almacena en \textbf{caché} la resposta a cada
consulta que realitza i guarda aquesta resposta final durant un temps
determinar (TTL = Time to Live).

Quan un solucionador o \emph{resolver} recursiu rep una consulta per a
una adreça IP que tingui al seu caché, pot proporcionar ràpidament la
resposta al caché al client sense comunicar-se amb cap altre servidor
DNS. Servir ràpidament respostes des del caché és molt probable si a) el
servidor DNS serveix a molts clients ob) el lloc web sol·licitat és molt
popular.

\hypertarget{desaventatges-del-dns-recursiu}{%
\subsection{\texorpdfstring{\textbf{Desaventatges del DNS
recursiu}}{Desaventatges del DNS recursiu}}\label{desaventatges-del-dns-recursiu}}

Desafortunadament, permetrà consultes de DNS recursives en servidors DNS
oberts creant una vulnerabilitat de seguretat, ja que aquesta
configuració pot permetre que els atacants portin a terme atacs
d'amplificació de DNS i d'envergament de caché de DNS .

\hypertarget{servidors-dns-recursius-i-atacs-damplificaciuxf3-de-dns}{%
\subsection{\texorpdfstring{\textbf{Servidors DNS recursius i atacs
d'amplificació de
DNS}}{Servidors DNS recursius i atacs d'amplificació de DNS}}\label{servidors-dns-recursius-i-atacs-damplificaciuxf3-de-dns}}

En un atac d'amplificació de DNS, un atacant sol utilitzar un grup de
màquines (que coneix com a botnet ) per enviar un gran volum de
consultes DNS mitjançant l'ús d'una adreça IP falsificada . Una direcció
IP falsificada és com una direcció de retorn falsa; l'atacant envia
sol·licituds des de la seva pròpia IP, però pide que les respostes van a
la víctima.

Per agravar l'atac, l'atacant també utilitza una tècnica trucada
amplificació, en la que la sol·licitud falsificada pide una resposta
molt llarga. El servei víctima rebrà una allau de respostes de DNS
llargues i no desitjades que poden interrompre o fins i tot fer els seus
servidors. Aquest és un tipus d' atac DDoS.

És com si un grup d'adolescents bromistas llamara a una pizzeria i
pidiera cada un una docena de pizzes. Al lloc de la seva pròpia direcció
per a la entrega, a la direcció d'un veí despres. A la víctima, que rep
una enorme quantitat de pizzes familiars que no ha fet cap comanda,
probablement li pertorbirà el dia.

Necessiteu un servidor DNS que accepteu consultes recursives per dur a
terme un tipus d'atac, ja que els paquets de DNS amplificats són
respostes a consultes de DNS recursius.

\hypertarget{servidors-dns-recursius-i-atacs-denvergament-de-cachuxe9-de-dns}{%
\subsection{\texorpdfstring{\textbf{Servidors DNS recursius i atacs
d'envergament de caché de
DNS}}{Servidors DNS recursius i atacs d'envergament de caché de DNS}}\label{servidors-dns-recursius-i-atacs-denvergament-de-cachuxe9-de-dns}}

En un atac d'enverinament de caché de DNS, quan un servidor DNS recursiu
sol·licita una adreça IP a un altre servidor DNS, un atacant intercepta
la sol·licitud i una resposta falsa, que sol·liciteu la direcció IP d'un
lloc web maliciós.

El servidor DNS recursiu no només enviava al client original aquesta
adreça IP, sinó que el servidor també guarda la resposta al seu caché.

Cal demanar usuari que sol·liciti una IP per al mateix nom de domini
serà enviat al lloc web maliciós.

Si es tracta d'un nom de domini i un solucionador de DNS famosos, aquest
atac podria arribar a afectar a milles d'usuaris.

En una consulta de DNS iterativa, el client pide directament la resposta
a cada servidor DNS.

Inclós si un atacant és capaç d'enviar una resposta falsificada a la
consulta, només afectarà a un únic client, el que no mereixi el temps de
l'atacant.

\hypertarget{configuraciuxf3-dns-cryptosec}{%
\section{\texorpdfstring{\textbf{Configuració DNS
CryptoSEC}}{Configuració DNS CryptoSEC}}\label{configuraciuxf3-dns-cryptosec}}

\hypertarget{installaciuxf3}{%
\subsection{\texorpdfstring{\textbf{Instal·lació}}{Instal·lació}}\label{installaciuxf3}}

S'instal·la amb la comanda \texttt{apt-get\ install\ bind9}, el fitxer
de configuració es troba a \texttt{/etc/bind}.

\hypertarget{el-servidor-dns-autoritatiu}{%
\subsection{\texorpdfstring{\textbf{El servidor DNS
Autoritatiu}}{El servidor DNS Autoritatiu}}\label{el-servidor-dns-autoritatiu}}

Tindrà els registres de la zona \textbf{``cryptosec.net''}. És un
servidor autoritari que rebrà les peticions DNS d'un forwarder.

\hypertarget{arxiu-de-dopcions-de-les-zones}{%
\subsubsection{\texorpdfstring{\textbf{Arxiu de d'opcions de les
zones}}{Arxiu de d'opcions de les zones}}\label{arxiu-de-dopcions-de-les-zones}}

El fitxer \texttt{/etc/bind/named.conf.options}.

\begin{Shaded}
\begin{Highlighting}[]
        \ExtensionTok{//}\NormalTok{ forwarders \{}
        \ExtensionTok{//}\NormalTok{      0.0.0.0}\KeywordTok{;}
        \ExtensionTok{//}\NormalTok{ \};}

        \ExtensionTok{/}\NormalTok{/========================================================================}
        \ExtensionTok{//}\NormalTok{ If BIND logs error messages about the root key being expired,}
        \ExtensionTok{//}\NormalTok{ you will need to update your keys.  See https://www.isc.org/bind-keys}
        \ExtensionTok{/}\NormalTok{/========================================================================}

        \ExtensionTok{dnssec-enable}\NormalTok{ yes}\KeywordTok{;}
        \ExtensionTok{dnssec-validation}\NormalTok{ yes}\KeywordTok{;}
        \ExtensionTok{dnssec-lookaside}\NormalTok{ auto}\KeywordTok{;}

        \ExtensionTok{listen-on-v6}\NormalTok{ \{ any}\KeywordTok{;}\NormalTok{ \};}
\NormalTok{\};}
\end{Highlighting}
\end{Shaded}

Conté:

\begin{enumerate}
\def\labelenumi{\arabic{enumi}.}
\item
  La declaració del directori on es guardaran els arxius de zona:
  \textbf{/etc/bind}
\item
  La declaració, per defecte desactivada, dels servidors de reenviament:
  secció forwarders \{\ldots{}\}
\end{enumerate}

Si no s'utilitzen forwarders, el servidor DNS anirà als servidors arrel
per iniciar les resolucions de les consultes que no estiguin en memòria
cau ni a cap de les seves zones. Quan s'utilitzen servidors de
reenviament es consultarà aquests servidors.

\hypertarget{arxiu-de-dades-per-especificar-la-zona}{%
\subsubsection{\texorpdfstring{\textbf{Arxiu de dades per especificar la
zona}}{Arxiu de dades per especificar la zona}}\label{arxiu-de-dades-per-especificar-la-zona}}

\textbf{És un servidor SOA, State of Authority}

El fitxer \texttt{/etc/bind/named.conf.default-zones}:

\begin{Shaded}
\begin{Highlighting}[]
\ExtensionTok{zone} \StringTok{"cryptosec.net"}\NormalTok{ \{}
        \BuiltInTok{type}\NormalTok{ master}\KeywordTok{;}
        \FunctionTok{file} \StringTok{"/etc/bind/db.cryptosec.net"}\KeywordTok{;}
\NormalTok{\};}
\end{Highlighting}
\end{Shaded}

Cada zona (directa o inversa) tindrà:

\begin{enumerate}
\def\labelenumi{\arabic{enumi}.}
\item
  La declaració amb la directiva zoneon s'indica el domini o l'adreça de
  xarxa a les zones inverses.
\item
  Una directiva typeindicant si és una zona mestra (escrita per
  l'administrador) o esclava (descarregada automàticament d'un servidor
  mestre).
\item
  Una directiva fileindicant el fitxer de respatller (que es trobarà a
  /var/cache/bind)
\end{enumerate}

\hypertarget{arxiu-de-dades-per-a-una-zona-directa-cryptosec.net}{%
\subsubsection{\texorpdfstring{\textbf{Arxiu de dades per a una zona
directa
``\emph{cryptosec.net}''}}{Arxiu de dades per a una zona directa ``cryptosec.net''}}\label{arxiu-de-dades-per-a-una-zona-directa-cryptosec.net}}

Cada zona necessita un fitxer de dades on desar els registres de la
zona. Per a una zona directa com \texttt{cryptosec.net} el fitxer de
zona pot ser \texttt{/etc/bind/db.cryptosec.net} i contenir:

\begin{Shaded}
\begin{Highlighting}[]
\VariableTok{$TTL}    \ExtensionTok{604800}
\ExtensionTok{@}\NormalTok{       IN      SOA     cryptosec.net. mail.cryptosec.net. (}
                              \ExtensionTok{2}         \KeywordTok{;} \ExtensionTok{Serial}
                         \ExtensionTok{604800}         \KeywordTok{;} \ExtensionTok{Refresh}
                          \ExtensionTok{86400}         \KeywordTok{;} \ExtensionTok{Retry}
                        \ExtensionTok{2419200}         \KeywordTok{;} \ExtensionTok{Expire}
                         \ExtensionTok{604800}\NormalTok{ )       ; }\ExtensionTok{Negative}\NormalTok{ Cache TTL}

\ExtensionTok{@}\NormalTok{       IN      NS      cryptosec.net.}
\ExtensionTok{@}\NormalTok{       IN      A       192.168.0.164}
\ExtensionTok{www}\NormalTok{     IN      CNAME   cryptosec.net.}
\NormalTok{;}\ExtensionTok{@}\NormalTok{      IN      A       10.200.243.164}
\NormalTok{;}\ExtensionTok{@}\NormalTok{      IN      A       192.168.31.164}
\VariableTok{$INCLUDE} \StringTok{"/etc/bind/keys/zsk/Kcryptosec.net.+007+53495.key"}
\VariableTok{$INCLUDE} \StringTok{"/etc/bind/keys/ksk/Kcryptosec.net.+007+07353.key"}
\end{Highlighting}
\end{Shaded}

\begin{quote}
NOTE: \$INCLUDE ``/etc/bind/keys/zsk/Kcryptosec.net.+007+53495.key'' /
\$INCLUDE ``/etc/bind/keys/ksk/Kcryptosec.net.+007+07353.key'' son lés
claus de \textbf{DNSSEC}. Vegeu la documentació de \textbf{DNSSEC}.
\end{quote}

En aquest arxiu de zona cal notar:

\begin{itemize}
\item
  El caràcter @equival al domini que estigui definint acabat en punt.
  Aquí \texttt{cryptosec.net.}
\item
  El camp \texttt{mail.cryptosec.net.} correspon al correu de contacte
  per indicar errors a la zona i s'interpreta com
  \texttt{cryptosec.net.}
\item
  És important incrementar el valor Serial cada cop que es fa una
  modificació.
\end{itemize}

\hypertarget{el-servidor-dns-forwarder}{%
\subsection{\texorpdfstring{\textbf{El servidor DNS
Forwarder}}{El servidor DNS Forwarder}}\label{el-servidor-dns-forwarder}}

És un servidor DNS que s'encarregarà d'encaminar les peticions DNS dels
seus clients al SOA.

\hypertarget{arxiu-de-dopcions-de-les-zones-1}{%
\subsubsection{\texorpdfstring{\textbf{Arxiu de d'opcions de les
zones}}{Arxiu de d'opcions de les zones}}\label{arxiu-de-dopcions-de-les-zones-1}}

El fitxer \texttt{/etc/bind/named.conf.options}.

\begin{Shaded}
\begin{Highlighting}[]
        \ExtensionTok{forwarders}\NormalTok{ \{}
        \ExtensionTok{//}\NormalTok{ CLASE}
                \ExtensionTok{10.200.243.164}\KeywordTok{;}
\NormalTok{        \};}
        \ExtensionTok{dnssec-validation}\NormalTok{ no}\KeywordTok{;}
        \ExtensionTok{querylog}\NormalTok{ yes}\KeywordTok{;}
        \ExtensionTok{recursion}\NormalTok{ yes}\KeywordTok{;}
\end{Highlighting}
\end{Shaded}

Conté:

\begin{enumerate}
\def\labelenumi{\arabic{enumi}.}
\item
  La declaració del directori on es guardaran els arxius de zona:
  \textbf{/etc/bind}
\item
  La declaració, està activada, reenviarà els paquets al \textbf{SOA}:
  secció forwarders \{\ldots{}\}
\end{enumerate}

\hypertarget{arxiu-de-dades-per-especificar-el-forwarding-a-la-zona-cryptosec.net}{%
\subsubsection{\texorpdfstring{\textbf{Arxiu de dades per especificar el
forwarding a la zona
``\emph{cryptosec.net}''}}{Arxiu de dades per especificar el forwarding a la zona ``cryptosec.net''}}\label{arxiu-de-dades-per-especificar-el-forwarding-a-la-zona-cryptosec.net}}

\textbf{És un servidor SOA, State of Authority}

El fitxer \texttt{/etc/bind/named.conf.default-zones}:

\begin{Shaded}
\begin{Highlighting}[]
\ExtensionTok{zone} \StringTok{"cryptosec.net"}\NormalTok{ \{}
        \BuiltInTok{type}\NormalTok{ forward}\KeywordTok{;}
\ExtensionTok{//}\NormalTok{ CLASSE}
        \ExtensionTok{forwarders}\NormalTok{ \{ 10.200.243.164}\KeywordTok{;}\NormalTok{ \};}
\NormalTok{\};}
\end{Highlighting}
\end{Shaded}

Cada zona (directa o inversa) tindrà:

\begin{enumerate}
\def\labelenumi{\arabic{enumi}.}
\item
  La declaració amb la directiva zoneon s'indica el domini o l'adreça de
  xarxa a les zones inverses.
\item
  Una directiva typeindicant si és una zona mestra (escrita per
  l'administrador) o esclava (descarregada automàticament d'un servidor
  mestre).
\item
  Una directiva fileindicant el fitxer de respatller (que es trobarà a
  /var/cache/bind)
\end{enumerate}

\hypertarget{comandes-de-verificaciuxf3}{%
\subsection{\texorpdfstring{\textbf{Comandes de
verificació}}{Comandes de verificació}}\label{comandes-de-verificaciuxf3}}

\begin{itemize}
\item
  \texttt{journalctl\ -u\ named\ -f\ \&}: Mostra els logs del servei
  Bind9 en temps real
\item
  \texttt{systemctl\ restart\ bind9}: Reinicia el Bind9.
\item
  \texttt{host\ cryptosec.net}: Petició per resoldre la zona,
  \textbf{cryptosec.net}, obtindrem la IP.
\item
  \texttt{nslookup\ cryptosec.net}: Petició per resoldre la zona,
  \textbf{cryptosec.net}, obtindrem la IP
\item
  \texttt{dig\ cryptosec.net}: Petició per resoldre la zona,
  \textbf{cryptosec.net}, obtindrem la IP
\item
  \texttt{systemd-resolve\ -\/-status}: Verificació del status actual
  del DNS.
\item
  \texttt{resolvectl\ query\ cryptosec.net}: Petició per resoldre la
  zona, \textbf{cryptosec.net}, obtindrem la IP
\end{itemize}

\hypertarget{el-client-dns}{%
\subsection{\texorpdfstring{\textbf{El client
DNS}}{El client DNS}}\label{el-client-dns}}

És el component que delega una consulta o consulta directament a
servidors DNS, en la cerca d'un registre DNS a la qual vol accedir.

Utilitza el fitxer /etc/resolv.conf com a configuració del
\emph{resolver}.

La seva funció és millorar el rendiment de les resolucions mitjançant
memòria cau. Quan una resolució provoca una fallada de memòria cau
s'utilitzarà el DNS extern del qual probablement s'haurà obtingut la IP
mitjançant una concessió DHCP.

\includegraphics{./tex2pdf.-0a4f409d8d6fb9b0/380d1bbd9e6b8545103cf8a27a7508fdf59365eb.jpg}

\includegraphics{./tex2pdf.-0a4f409d8d6fb9b0/38cd9b267193da1c285e6fee5b20196a24b86ed4.jpg}

\hypertarget{resoluciuxf3-de-noms-al-client}{%
\subsection{\texorpdfstring{\textbf{Resolució de noms al
client}}{Resolució de noms al client}}\label{resoluciuxf3-de-noms-al-client}}

Quan un client vol comunicarśe amb una altra de la que només conéix el
FQDN (\emph{Fully Qualified Domain Name} = www.cryptosec.net), primer
seria obtenir l'adreça IP amb el nom de domini. Després fa el request
HTTP per poder accedir-hi.

Podem utilitzar el fitxer \textbf{/etc/hosts} com a resolver local o be
als servidors DNS establertes en \textbf{/etc/resolv.conf}

\hypertarget{exemple-de-etchosts}{%
\subsubsection{\texorpdfstring{\textbf{Exemple de
/etc/hosts}}{Exemple de /etc/hosts}}\label{exemple-de-etchosts}}

\begin{Shaded}
\begin{Highlighting}[]
\ExtensionTok{127.0.0.1}\NormalTok{ localhost }
\ExtensionTok{82.151.203.129}\NormalTok{ iespuigcastellar.xeill.net }
\ExtensionTok{145.97.39.155}\NormalTok{ es.wikipedia.org }
\ExtensionTok{192.168.3.1}\NormalTok{ cryptosec.net}
\end{Highlighting}
\end{Shaded}

\hypertarget{exemple-de-etcresolv.conf}{%
\subsubsection{\texorpdfstring{\textbf{Exemple de
/etc/resolv.conf}}{Exemple de /etc/resolv.conf}}\label{exemple-de-etcresolv.conf}}

Quan s'utilitza el client DNS per obtenir l'adreça IP d'un nom de
domini, cal examinar el fitxer de configuració \textbf{/etc/resolv.conf}
per obtenir:

\begin{itemize}
\item
  La llista de servidors DNS a utilitzar (un per línia precedit per la
  directiva nameserver)
\item
  El domini a utilitzar per a les consultes que no són un FQDN indicat
  per la directiva search
\end{itemize}

\begin{Shaded}
\begin{Highlighting}[]
\CommentTok{# Recerca de cryptosec.net }
\ExtensionTok{nameserver}\NormalTok{ 192.168.3.1}
\ExtensionTok{nameserver}\NormalTok{ 10.200.244.10}
\ExtensionTok{search}\NormalTok{ cryptosec.net}
\end{Highlighting}
\end{Shaded}

\hypertarget{el-servei-systemd-resolved-i-la-comanda-resolvectl}{%
\section{\texorpdfstring{\textbf{El servei systemd-resolved i la comanda
resolvectl}}{El servei systemd-resolved i la comanda resolvectl}}\label{el-servei-systemd-resolved-i-la-comanda-resolvectl}}

La majoria de les distribucions actuals de GNU/Linux utilitzen
systemdaixí que solen executar el servei systemd-resolvedcom a stub DNS
local de la màquina. L'avantatge dutilitzar \textbf{systemd-resolved} és
que les aplicacions trobaran un millor rendiment gràcies a la seva
memòria cau.

El fitxer \textbf{/etc/resolv.conf} pot ser:

\begin{Shaded}
\begin{Highlighting}[]
\CommentTok{# See man:systemd-resolved.service(8) for details about the supported modes of}
\CommentTok{# operation for /etc/resolv.conf.}

\ExtensionTok{nameserver}\NormalTok{ 127.0.0.53}
\ExtensionTok{options}\NormalTok{ edns0 trust-ad}
\ExtensionTok{search}\NormalTok{ cryptosec.net}
\end{Highlighting}
\end{Shaded}

Aquí es pot observar:

\begin{itemize}
\item
  El comentari adverteix que és un fitxer generat per
  \textbf{systemd-resolved}.
\item
  Com a servidor DNS s'ha configurat l'adreça 127.0.0.53 que només és
  accessible des del propi equip.
\end{itemize}

L'ordre \texttt{resolvectl} permet:

\begin{itemize}
\item
  Mostrar informació sobre la configuració: \textbf{resolvectl status}
\item
  Mostra estadístiques sobre els encerts de memòria cau:
  \textbf{resolvectl statistics}
\item
  Mostra els DNS utilitzats: \textbf{resolvectl dns}
\item
  Fer resolucions DNS: resolvectl query \textbf{cryptosec.net}
\end{itemize}

\includegraphics{./tex2pdf.-0a4f409d8d6fb9b0/62e1cdb7fb8dcb3748a3c5c96634168cf1d3f73a.jpg}

\hypertarget{com-donar-suport-a-consultes-de-dns-ruxe0pides-i-segures}{%
\section{\texorpdfstring{\textbf{Com donar suport a consultes de DNS
ràpides i
segures}}{Com donar suport a consultes de DNS ràpides i segures}}\label{com-donar-suport-a-consultes-de-dns-ruxe0pides-i-segures}}

La solució es DNSSEC.

Consulteu la documentació més extensa i adherida de \textbf{DNSSEC}.

\includegraphics{./tex2pdf.-0a4f409d8d6fb9b0/229116373a21fc0addbdb8caacb00b44ce90d3f2.jpg}

\hypertarget{glossari-de-termes-de-les-configuracions-de-bind9}{%
\section{\texorpdfstring{\textbf{Glossari de termes de les
configuracions de
BIND9}}{Glossari de termes de les configuracions de BIND9}}\label{glossari-de-termes-de-les-configuracions-de-bind9}}

\texttt{A} = stableix la correspondència d'un nom a una adreça IP.

\texttt{CNAME} = Estableix un àlies per a un nom.

\texttt{MX} = Especifica un servidor d'intercanvi de correu.

\texttt{NS} = Especifica un servidor de noms per al domini.

\texttt{PTR} = Estableix la correspondència d'una adreça IP a un nom.

\texttt{SOA} = Declara el servidor de noms que té més autoritat per a la
zona.

\hypertarget{glossari-de-termes-de-tipus-de-servidors-de-bind9}{%
\section{\texorpdfstring{\textbf{Glossari de termes de tipus de
servidors de
BIND9}}{Glossari de termes de tipus de servidors de BIND9}}\label{glossari-de-termes-de-tipus-de-servidors-de-bind9}}

\texttt{Recursive} = El servidor que realitza diverses consultes fins a
obtenir una resposta completa.

\texttt{Secondary} = El servidor DNS que rep la base de dades de zona
d'un altre servidor.

\texttt{Forward} = El servidor que reenvia les consultes a altres
servidors.

\texttt{Authoritative} = El servidor que proporciona respostes sobre les
zones que té configurades.

\texttt{Primary} = El servidor DNS que conté els fitxers de zona (la
base de dades original).

\hypertarget{glossari-de-termes-seongs-cada-camp-del-soa-amb-la-seva-funciuxf3-bind9}{%
\section{\texorpdfstring{\textbf{Glossari de termes seongs cada camp del
SOA amb la seva funció
(Bind9)}}{Glossari de termes seongs cada camp del SOA amb la seva funció (Bind9)}}\label{glossari-de-termes-seongs-cada-camp-del-soa-amb-la-seva-funciuxf3-bind9}}

\texttt{time-to-retry} = Temps que un servidor secundari deixa passar
abans de tornar a intentar una transferència de zona.

\texttt{serial-number} = Utilitzat pels servidors secundaris per a
detectar els canvis a la base de dades.

\texttt{nxdomain-ttl} = El temps que els solucionadors han de guardar a
la memòria cau una resposta de domini o host inexistent..

\texttt{time-to-expire} = Si passat aquest temps, un servidor secundari
no aconsegueix realitzar la transferència de zona, ha de deixar de
considerar-la vàlida.

\hypertarget{exemples-bind9-configuracions}{%
\section{\texorpdfstring{\textbf{Exemples BIND9
(Configuracions)}}{Exemples BIND9 (Configuracions)}}\label{exemples-bind9-configuracions}}

\hypertarget{glue-record}{%
\subsection{\texorpdfstring{\textbf{GLUE
RECORD}}{GLUE RECORD}}\label{glue-record}}

\begin{Shaded}
\begin{Highlighting}[]
\CommentTok{# /etc/bind/named.conf.default-zones}

\ExtensionTok{zone} \StringTok{"zone1.com"}\NormalTok{ \{}
    \BuiltInTok{type}\NormalTok{ primary}\KeywordTok{;}
    \FunctionTok{file} \StringTok{"/etc/bind/db.zone1.com"}\KeywordTok{;}
\NormalTok{\};}
\end{Highlighting}
\end{Shaded}

\begin{Shaded}
\begin{Highlighting}[]
\CommentTok{# /etc/bind/db.zone1.com}

\ExtensionTok{@}\NormalTok{   SOA ns  admin   1700010100 20m 3m 2w 1h}
    \ExtensionTok{NS}\NormalTok{  ns}
    \ExtensionTok{A}\NormalTok{   1.1.1.1}
\end{Highlighting}
\end{Shaded}

Què respondra el servidor a la següent consulta?

\begin{Shaded}
\begin{Highlighting}[]
\ExtensionTok{user@debian}\NormalTok{:~$ host zone1.com}
\end{Highlighting}
\end{Shaded}

No es pot carregar la Zona ja que falta el GLUE RECORD:

ns A 2.2.2.2

La resposta correcta és: \texttt{SERVFAIL}

\hypertarget{generate}{%
\subsection{\texorpdfstring{\textbf{\$GENERATE}}{\$GENERATE}}\label{generate}}

\begin{Shaded}
\begin{Highlighting}[]
\CommentTok{# /etc/bind/named.conf.default-zones}

\ExtensionTok{zone} \StringTok{"server.tld"}\NormalTok{ \{}
    \BuiltInTok{type}\NormalTok{ primary}\KeywordTok{;}
    \FunctionTok{file} \StringTok{"/etc/bind/db.server.tld"}\KeywordTok{;}
\NormalTok{\};}
\end{Highlighting}
\end{Shaded}

\begin{Shaded}
\begin{Highlighting}[]
\CommentTok{# /etc/bind/db.server.tld}

\ExtensionTok{@}\NormalTok{        SOA    ns    admin    1700010100 20m 3m 2w 1h}
         \ExtensionTok{NS}\NormalTok{     ns}
\ExtensionTok{ns}\NormalTok{       A      192.168.122.1}
\VariableTok{$GENERATE} \ExtensionTok{1-4}\NormalTok{ database$ A 1.1.$.1}
\end{Highlighting}
\end{Shaded}

\begin{Shaded}
\begin{Highlighting}[]
\ExtensionTok{user@debian}\NormalTok{:~$ host database1.server.tld}
\ExtensionTok{database1.server.tld}\NormalTok{ has address 1.1.1.1}
\ExtensionTok{user@debian}\NormalTok{:~$ host database2.server.tld}
\ExtensionTok{database2.server.tld}\NormalTok{ has address 1.1.2.1}
\ExtensionTok{user@debian}\NormalTok{:~$ host database3.server.tld}
\ExtensionTok{database3.server.tld}\NormalTok{ has address 1.1.3.1}
\ExtensionTok{user@debian}\NormalTok{:~$ host database4.server.tld}
\ExtensionTok{database4.server.tld}\NormalTok{ has address 1.1.4.1}
\ExtensionTok{user@debian}\NormalTok{:~$ host database5.server.tld}
\ExtensionTok{Host}\NormalTok{ database5.server.tld not found: 3(NXDOMAIN)}
\end{Highlighting}
\end{Shaded}

\hypertarget{resoluciuxf3-inversa}{%
\subsection{\texorpdfstring{\textbf{Resolució
inversa}}{Resolució inversa}}\label{resoluciuxf3-inversa}}

\begin{Shaded}
\begin{Highlighting}[]
\CommentTok{# /etc/bind/named.conf.default-zones}

\ExtensionTok{zone} \StringTok{"5.43.IN-ADDR.ARPA"}\NormalTok{ \{}
    \BuiltInTok{type}\NormalTok{ primary}\KeywordTok{;}
    \FunctionTok{file} \StringTok{"/etc/bind/db.5.43.IN-ADDR.ARPA"}\KeywordTok{;}
\NormalTok{\};}
\end{Highlighting}
\end{Shaded}

\begin{Shaded}
\begin{Highlighting}[]
\CommentTok{# /etc/bind/db.5.43.IN-ADDR.ARPA}

\ExtensionTok{@}\NormalTok{       SOA ns  admin   1700010100 20m 3m 2w 1h}
        \ExtensionTok{NS}\NormalTok{  ns}
\ExtensionTok{ns}\NormalTok{      A   192.168.122.1}
\ExtensionTok{78.202}\NormalTok{      PTR host.tld.}
\end{Highlighting}
\end{Shaded}

Quina adreça IP es resoldrà inversament al nom host.tld?
\texttt{43.5.202.78}

\hypertarget{bibliografia}{%
\section{\texorpdfstring{\textbf{Bibliografia}}{Bibliografia}}\label{bibliografia}}

\begin{itemize}
\tightlist
\item
  https://www.cloudflare.com/learning/dns/dns-over-tls/
\item
  https://www.cloudflare.com/es-es/learning/dns/what-is-dns/
\item
  https://elpuig.xeill.net/Members/vcarceler/c1/didactica/apuntes/ud4/na8
\item
  https://www.cloudflare.com/learning/dns/what-is-dns/
\item
  https://www.digival.es/blog/que-son-las-dns-y-para-que-sirven/
\item
  https://www.hostinger.es/tutoriales/que-es-dns
\item
  https://www.webempresa.com/hosting/que-son-dns.html
\item
  https://dinahosting.com/ayuda/que-es-un-servidor-dns/
\item
  https://es.wikipedia.org/wiki/Sistema\_de\_nombres\_de\_dominio
\item
  https://www.csuc.cat/ca/serveis/secundaris-i-repliques-de-dns
\item
  https://ca.eyewated.com/que-es-dns-domain-name-system/
\item
  https://ca.theastrologypage.com/dns-record
\item
  http://acacha.org/mediawiki/Servidor\_DNS\#.Yoj9sKjP1PY
\end{itemize}

\end{document}
