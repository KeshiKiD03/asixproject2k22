\PassOptionsToPackage{unicode=true}{hyperref} % options for packages loaded elsewhere
\PassOptionsToPackage{hyphens}{url}
%
\documentclass[]{article}
\usepackage{lmodern}
\usepackage{amssymb,amsmath}
\usepackage{ifxetex,ifluatex}
\usepackage{fixltx2e} % provides \textsubscript
\ifnum 0\ifxetex 1\fi\ifluatex 1\fi=0 % if pdftex
  \usepackage[T1]{fontenc}
  \usepackage[utf8]{inputenc}
  \usepackage{textcomp} % provides euro and other symbols
\else % if luatex or xelatex
  \usepackage{unicode-math}
  \defaultfontfeatures{Ligatures=TeX,Scale=MatchLowercase}
\fi
% use upquote if available, for straight quotes in verbatim environments
\IfFileExists{upquote.sty}{\usepackage{upquote}}{}
% use microtype if available
\IfFileExists{microtype.sty}{%
\usepackage[]{microtype}
\UseMicrotypeSet[protrusion]{basicmath} % disable protrusion for tt fonts
}{}
\IfFileExists{parskip.sty}{%
\usepackage{parskip}
}{% else
\setlength{\parindent}{0pt}
\setlength{\parskip}{6pt plus 2pt minus 1pt}
}
\usepackage{hyperref}
\hypersetup{
            pdfborder={0 0 0},
            breaklinks=true}
\urlstyle{same}  % don't use monospace font for urls
\usepackage{graphicx,grffile}
\makeatletter
\def\maxwidth{\ifdim\Gin@nat@width>\linewidth\linewidth\else\Gin@nat@width\fi}
\def\maxheight{\ifdim\Gin@nat@height>\textheight\textheight\else\Gin@nat@height\fi}
\makeatother
% Scale images if necessary, so that they will not overflow the page
% margins by default, and it is still possible to overwrite the defaults
% using explicit options in \includegraphics[width, height, ...]{}
\setkeys{Gin}{width=\maxwidth,height=\maxheight,keepaspectratio}
\setlength{\emergencystretch}{3em}  % prevent overfull lines
\providecommand{\tightlist}{%
  \setlength{\itemsep}{0pt}\setlength{\parskip}{0pt}}
\setcounter{secnumdepth}{0}
% Redefines (sub)paragraphs to behave more like sections
\ifx\paragraph\undefined\else
\let\oldparagraph\paragraph
\renewcommand{\paragraph}[1]{\oldparagraph{#1}\mbox{}}
\fi
\ifx\subparagraph\undefined\else
\let\oldsubparagraph\subparagraph
\renewcommand{\subparagraph}[1]{\oldsubparagraph{#1}\mbox{}}
\fi

% set default figure placement to htbp
\makeatletter
\def\fps@figure{htbp}
\makeatother


\date{}

\begin{document}

\hypertarget{projecte-asix-2k22}{%
\section{\texorpdfstring{\textbf{Projecte ASIX
2k22}}{Projecte ASIX 2k22}}\label{projecte-asix-2k22}}

\hypertarget{escola-del-treball}{%
\subsection{\texorpdfstring{\textbf{Escola Del
Treball}}{Escola Del Treball}}\label{escola-del-treball}}

\hypertarget{hisx-2021-2022}{%
\subsubsection{\texorpdfstring{\textbf{2HISX
2021-2022}}{2HISX 2021-2022}}\label{hisx-2021-2022}}

\hypertarget{aaron-andal-cristian-condolo}{%
\subsubsection{\texorpdfstring{\textbf{Aaron Andal \& Cristian
Condolo}}{Aaron Andal \& Cristian Condolo}}\label{aaron-andal-cristian-condolo}}

\hypertarget{ciberseguretat-careful-where-you-step-uxfe0f}{%
\section{\texorpdfstring{\textbf{Ciberseguretat}: ``\emph{Careful where
you step}'' 🕵️
🔎}{Ciberseguretat: ``Careful where you step'' 🕵️ 🔎}}\label{ciberseguretat-careful-where-you-step-uxfe0f}}

\includegraphics{./tex2pdf.-bfb80653790035d4/c591f12351b9a0389258feee103fc702ad34cdff.jpg}

\hypertarget{index}{%
\section{Index}\label{index}}

\begin{itemize}
\item
  \textbf{OpenVAS}: \protect\hyperlink{descripciuxf3biografia}{README}
\item
  \textbf{Practica}: \protect\hyperlink{practica}{README}
\item
  \textbf{Bibliografia}: \protect\hyperlink{bibliografia}{README}
\end{itemize}

\begin{quote}
\textbf{NOTA: per desgràcia, no hem pogut implementar a dins. Degut a
quant estava tot instal·lat no trovaba les xarxes i els hosts.}
\end{quote}

\hypertarget{openvas-open-vulnerability-assessment-system}{%
\subsection{\texorpdfstring{\textbf{OpenVAS}: Open Vulnerability
Assessment
System}{OpenVAS: Open Vulnerability Assessment System}}\label{openvas-open-vulnerability-assessment-system}}

Es un escàner de vulnerabilitats amb totes les funcions. Les seves
capacitats inclouen proves no autenticades i autenticades, diversos
protocols industrials i d'Internet d'alt i baix nivell, ajust de
rendiment per a exploracions a gran escala i un potent llenguatge de
programació intern per implementar qualsevol tipus de prova de
vulnerabilitat. L'escàner obté les proves per detectar vulnerabilitats a
partir d'un canal que té un llarg historial i actualitzacions diàries.

OpenVAS ha estat desenvolupat i impulsat per l'empresa Greenbone
Networks des de l'any 2006. Com a part de la família de productes de
gestió de vulnerabilitats comercials Greenbone Enterprise Appliance,
l'escàner forma Greenbone Vulnerability Management juntament amb altres
mòduls de codi obert.

\includegraphics{./tex2pdf.-bfb80653790035d4/a57859b9abf365574ec0ef308ba22aeb3026d508.png}

Ens va permetre escanejar objectius tan dispositius mòbils, dispositius
de xarxa, PC, etc. allò que sigui que estigui connectada a la nostra
xarxa. Amb el fi d'aconseguir possibles vulnerabilitats que tinguin
aquestes hosts i per poder fer dues coses: - Per una banda, si som
l'atacant o l'auditor, intentar explotar-les. - I si estem a l'equip de
defensa, intentar defensar-los i tancar-los correctament.

\includegraphics{./tex2pdf.-bfb80653790035d4/4aca869a7aa825d77f113c00b58666bf2a2000c4.png}

Dins del panl de monitoritzacio del OpenVAS podem veure les xarxes,
hosts o un grup d'IPs per poder escanjer a dispositius de xarxa, a
dispositius mobils, a servidors, a PC, a aplicacion, un munt de coses.

\includegraphics{./tex2pdf.-bfb80653790035d4/65889ff882e9c8a5d9480011592803151b0b1cec.png}

\hypertarget{practica}{%
\subsection{\texorpdfstring{\textbf{Practica}}{Practica}}\label{practica}}

\begin{enumerate}
\def\labelenumi{\arabic{enumi}.}
\tightlist
\item
  Actulitzar el sistema (pot trigar una estona!).
\end{enumerate}

\texttt{sudo\ apt\ update\ -y\ \&\&\ sudo\ apt\ -disupgrade\ -y}

\begin{enumerate}
\def\labelenumi{\arabic{enumi}.}
\setcounter{enumi}{1}
\tightlist
\item
  Aque ja si, instal·lar el paquet OpenVAS.
\end{enumerate}

\texttt{sudo\ apt\ install\ openvas\ -y}

\begin{enumerate}
\def\labelenumi{\arabic{enumi}.}
\setcounter{enumi}{2}
\tightlist
\item
  Ara passem a lo mes aburrit, esperar. Instal·lem l'aplicacio, per aixo
  necessitar descarregar totes les firmes per poder detectar
  vulnerabilitats que qualsevol sistema per exemple apache2, windows,
  \ldots{} En resum que trigarar un mun de hores. En el nostre cas va
  tarda 1 hora i mig . En un altre exemple va trigar 3 hores.
\end{enumerate}

\texttt{sudo\ gmv\ setup}

\includegraphics{./tex2pdf.-bfb80653790035d4/c5e34b5b01c5dda69272e8562db3231ee5ef3a44.png}

\begin{enumerate}
\def\labelenumi{\arabic{enumi}.}
\setcounter{enumi}{3}
\tightlist
\item
  Un cop acabat l'instal·lacio ens donara un nom d'usuari i un password
  per poder entrar al panel del OpenVAS. Es important guardar-ho en un
  lloc segur.
\end{enumerate}

\begin{verbatim}
admin
aa6f95ca-9641-47f4-bd7d-7a5c5a56b934
\end{verbatim}

\begin{enumerate}
\def\labelenumi{\arabic{enumi}.}
\setcounter{enumi}{4}
\tightlist
\item
  Primer inicem el openvas. En cas de que surti \texttt{Failed} el podem
  resoldre amb un \texttt{restart} o en aquest cas es un \texttt{stop} i
  un \texttt{star} de nou.
\end{enumerate}

\texttt{sudo\ gvm-start}

\includegraphics{./tex2pdf.-bfb80653790035d4/b1701b60bfc3dc3834fd2c74eea1f9eff1f518a7.png}
\includegraphics{./tex2pdf.-bfb80653790035d4/deca543632f19f76eb3543bd9edafaf5e079adfc.png}

\begin{enumerate}
\def\labelenumi{\arabic{enumi}.}
\setcounter{enumi}{5}
\tightlist
\item
  Quant el servidor s'engega, ja ens obre un navegador. Nomes queda
  aceptar el certificats i iniciar sessio al OpenVAS. Ja podem observer
  i escanejar els dispositus/hosts/IPs de la nostra xarxa i d'altres
  xarxes.
\end{enumerate}

\includegraphics{./tex2pdf.-bfb80653790035d4/b57f74e06a97e0529f42a3ff41b21dbb7fa9a5bb.png}
\includegraphics{./tex2pdf.-bfb80653790035d4/528e254d06b08a9a55c7670d911fff799b583882.png}
\includegraphics{./tex2pdf.-bfb80653790035d4/e94d8a9b3289990e2402238e769d37a246acab16.png}

\hypertarget{bibliografia}{%
\subsection{\texorpdfstring{\textbf{Bibliografia}}{Bibliografia}}\label{bibliografia}}

\begin{itemize}
\tightlist
\item
  https://www.youtube.com/watch?v=Sf9LKyCpgPc
\end{itemize}

\end{document}
